\documentclass[10pt, a4paper]{moderncv}

\moderncvtheme[blue]{classic}

\usepackage[scale=0.8]{geometry}
\recomputelengths

\firstname{Alberto}
\familyname{Oporto Ames}
\email{alberto.oporto@utec.edu.pe}
\homepage{github.com/otreblan}
\title{Estudiante de Ciencia de la Computación}
\mobile{915365499}
\address{Santiago de Surco, Av Las Gaviotas 2110, E13, 703}

\begin{document}

\maketitle
\section{Educación}
\cventry{2018-Presente}{Ciencia de la Computación}{Universidad de Ingeniería y Tecnología - UTEC}{Lima}{Perú}{Octavo ciclo}

\section{Idiomas}
\cvlanguage{Español}{Nativo}{}
\cvlanguage{Inglés}{Intermedio}{}
\cvlanguage{Japonés}{Básico}{JLPT N5}

%\section{Experiencia}

\section{Habilidades}
\subsection{Lenguajes de programación}
\cvlistitem{Bash}
\cvlistitem{C/C++}
\cvlistitem{C\#}
\cvlistitem{Java}
\cvlistitem{Python}
\cvlistitem{R}

\subsection{Extra}
\cvlistitem{Excel}
\cvlistitem{Git}
\cvlistitem{Linux}
\cvlistitem{SQL}

\section{Proyectos}
\cventry{2021}{Modelo predictor}{\httplink{github.com/cs2901-2021-1/project/tree/python}}{}{}
{
	Obtiene la cantidad esperada de matrículas de un curso para un
	determinado ciclo.
	A partir de la información de los estudiantes habilitados en
	una base de datos Oracle.
}
\cventry{2021}{Compilador}{\httplink{github.com/otreblan/cce}}{}{}
{Un compilador hecho en C++, C, Bison y Flex.}
\cventry{2019}{Juego}{\httplink{github.com/L603/SpaceClones}}{}{}
{Un clon de Space Invaders hecho en C++.}
\cventry{2019-Presente}{Arch User Repository maintainer}{\httplink{aur.archlinux.org/account/otreblan}}{}{}
{
	Me encargo de actualizar, parchear y crear paquetes para la distribución
	Arch Linux.
}

\end{document}
