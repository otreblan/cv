\documentclass[10pt, a4paper]{moderncv}

\moderncvtheme[blue]{classic}

\usepackage[scale=0.8]{geometry}
\recomputelengths

\firstname{Alberto}
\familyname{Oporto Ames}
\email{alberto.oporto@utec.edu.pe}
\homepage{github.com/otreblan}
%\title{Estudiante de Ciencia de la Computación}
\mobile{915365499}
\address{Santiago de Surco, Av Las Gaviotas 2110, E13, 703}
\extrainfo{\linkedinsocialsymbol \httplink{linkedin.com/in/otreblan}}

\begin{document}

\maketitle
\section{Presentación}
Estudio Ciencia de la Computación en la Universidad de Ingeniería y Tecnología.
%Con un promedio ponderado acumulado de 15.07/20.
Estoy interesado en proyectos open source, especialmente en Linux.
%Además sé Inglés y Japonés (JLPT N5).

\section{Educación}
\cventry{2018-Presente}{Ciencia de la Computación}{Universidad de Ingeniería y Tecnología - UTEC}{Lima}{Perú}{}

%\subsection{Cursos llevados}
%\cvlistitem{aaa}

\section{Idiomas}
%\cvlanguage{Español}{Nativo}{}
\cvlanguage{Inglés}{Intermedio}{}
\cvlanguage{Japonés}{Básico}{JLPT N5}

%\section{Experiencia}

\section{Lenguajes de programación}
%\cvlistitem{Excel}
%\cvlistitem{Linux}
\cvlistitem{Bash}
\cvlistitem{C/C++}
\cvlistitem{C\#}
\cvlistitem{Git}
\cvlistitem{Java}
\cvlistitem{Python}
\cvlistitem{R}
\cvlistitem{SQL}
\cvlistitem{\LaTeX}

\section{Proyectos}
\cventry{2021}{Modelo predictor de matrículas}{\httplink{github.com/cs2901-2021-1/project/tree/python}}{}{}
{
	UTEC.
	Obtiene la cantidad esperada de matrículas de un curso para un
	determinado ciclo.
	A partir de la información de los estudiantes habilitados en
	una base de datos Oracle.
	Desarrollado con Tensorflow en Python.
}
\cventry{2021}{Compilador en C/C++ con semántica reducida}{\httplink{github.com/otreblan/cce}}{}{}
{
	UTEC.
	Un compilador de una versión reducida de C en español.
	Con propósito educativo.
	Hecho en C++, C, Bison y Flex.
}
\cventry{2020}{Asistencia a docente}{}{}{}
{
	Apoyo en el uso de herramientas virtuales a los docentes.
	% Cómo parte de los requisitos para mantener la beca $BECA.
}
\cventry{2019}{Videojuego}{\httplink{github.com/L603/SpaceClones}}{}{}
{
	%Un clon de Space Invaders hecho en C++.
	Desarrollo de un videojuego basado en Space Invaders, hecho en C++.
}

\section{Voluntariado}
\cventry{2019-Presente}{Arch User Repository maintainer}{\httplink{aur.archlinux.org/account/otreblan}}{}{}
{
	Me encargo de actualizar, parchear y crear paquetes para la distribución
	Arch Linux.
}
%\cvlistitem{\textbf{Giara} Cliente de Reddit basado en Python y GTK.}
%\cvlistitem{\textbf{Legendary} Cliente y librería de Epic Games en Python.}

%\section{Certificaciones}
% TODO: Linkedin

\end{document}
