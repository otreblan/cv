\documentclass[10pt, a4paper]{moderncv}

\moderncvtheme[blue]{classic}

\usepackage[scale=0.81]{geometry}
\recomputelengths

\firstname{Alberto}
\familyname{Oporto}
\email{otreblain@gmail.com}
\title{Software engineer}
\mobile{+51 915365499}
\address{Lima, Perú}
\extrainfo{
	\linkedinsocialsymbol \httplink{linkedin.com/in/otreblan}\\
	\githubsocialsymbol \httplink{github.com/otreblan}
}

\begin{document}

\maketitle
% Ajustar la presentación de acuerdo a los requerimientos de la oferta.
% 4 líneas
%\section{Presentación}
%Estudio Ciencia de la Computación en UTEC.
%Pertenezco al tercio superior de mi carrera.
%Tengo interés en la tecnología y en desarrollar continuamente mis competencias tecnológicas, especialmente en proyectos Open Source.
%%Soy una persona autodidacta en el uso de nuevas herramientas, analizo y resuelvo problemas efectivamente.
%Además, muestro escucha activa y apertura ante cualquier feedback.

% TODO: Me falta un objetivo

\section{Experience}
% TODO: Potencial
\cventry{05/2023 - Present}{Software Engineer}{TITANIUS CONSULTING S.R.L.}{Lima}{Perú}{
	\begin{itemize}
		\item Developed backend microservices in multiple languages.
		\item Containerized CSR frontends and Java, C\# and Node based backends for local development and deployment.
		\item Automated SSL certificate renewal with Caddy and Let's Encrypt.
		\item Wrote Gitlab CI pipelines for automatic building, testing and deployment on self hosted Linux VPS.
	\end{itemize}
}
\cventry{02/2023 - 04/2023}{Programmer Trainee}{HEINZ - GLAS PERU S.A.C.}{Lima}{Perú}{
	\begin{itemize}
		\item Developed an internal MVC WebApp with C\# ASP.NET for production plant automatization.
		\item Deployed a self hosted Gitea instance on a legacy Windows Server.
		\item Wrote shell scripts to interface with legacy Linux systems.
	\end{itemize}
%Desarrollé una aplicación en C\# .NET con ASP.NET para la automatización de procesos de la planta de producción. Además escribí scripts en Bash para obtener información desde los servidores Linux.
}

\section{Education}
\cventry{2018 - 2024}{Bachelor of Computer Science}{Universidad de Ingeniería y Tecnología - UTEC}{Lima}{Perú}{}

\section{Languages}
\cvlanguage{Spanish}{Native}{}
\cvlanguage{English}{TOEFL ITP C1}{}
\cvlanguage{Japanese}{JLPT N3}{}

\section{Programming languages}
\cvlistitem{C/C++}
\cvlistitem{Java}
\cvlistitem{Typescript}
\cvlistitem{Bash}
\cvlistitem{C\#}
\cvlistitem{Python}

\section{Tech skills}
\cvlistitem{Linux}
\cvlistitem{CI/CD}
\cvlistitem{Docker / Docker compose}
\cvlistitem{Git / Github / Gitlab / Gitea}
\cvlistitem{SQL}

\section{Projects}
\cventry{}{Arch User Repository maintainer}{\httplink{aur.archlinux.org/account/otreblan}}{}{}
{
	I am in charge of packaging, updating and patching apps for the Arch User Repository.
}
\cventry{}{IOT UV sensor}{\httplink{github.com/otreblan/esp8266-uv}}{}{}
{
	%Con un esp8266 y un sensor de luz uv medimos la radiación a lo largo del día. Programado en C++ con Espressif RTOS.
	My team and I utilized an esp8266, a 3d printed case and an uv light sensor to log solar radiantion. I coded the firmware with C++ and Espressif RTOS.
}
%\cventry{}{Mini supercomputer}{\httplink{github.com/otreblan/mini-cluster}}{}{}
%{
%	My team and I integrated multiple Raspberry Pis into a HPC cluster.
%	Then we tested it with OpenMPI.
%	%Mi equipo y yo configuramos 7 Raspberry Pis para crear un mini cluster HPC.
%	%Luego ejecutamos pruebas de escalabilidad con OpenMPI.
%}
\cventry{}{Game engine}{\httplink{github.com/otreblan/vulkan-hello}}{}{}
{
	Implemented with C++ and Vulkan a game engine prototype based on Entity Component System.
	%Programé con C++ y Vulkan un motor de videojuegos con un paradigma basado en datos ().
	%Soporta físicas en tiempo real, input del usuario, importación de modelos 3D y renderización rasterizada.
}
%\cventry{2021}{Compilador en C/C++ con semántica reducida}{\httplink{github.com/otreblan/cce}}{}{}
%{
%	Programé un compilador de una versión reducida de C en español.
%	Con propósito educativo.
%	Usé C++, C, Bison y Flex.
%}
%\cventry{2019}{Videojuego}{\httplink{github.com/L603/SpaceClones}}{}{}
%{
%	Guié a mi equipo en el desarrollo de un videojuego basado en Space Invaders,
%	hecho en C++; por lo cual recibimos un reconocimiento por nuestro trabajo destacado.
%}

%\section{Voluntariado}

%\section{Certificaciones}
% TODO: Linkedin

\end{document}
