\documentclass[11pt, a4paper]{moderncv}
%\usepackage{main}

\moderncvtheme[blue]{classic}

\usepackage[scale=0.8]{geometry}
\recomputelengths

\firstname{Alberto}
\familyname{Oporto Ames}
\email{alberto.oporto@utec.edu.pe}
\homepage{github.com/otreblan}
\title{Estudiante de Ciencia de la Computación}
\mobile{915365499}

\begin{document}

\maketitle
\section{Educación}
\cventry{2018-2023}{Ciencia de la Computación}{Universidad de Ingeniería y Tecnología - UTEC}{Lima}{Perú}{Octavo ciclo}

\section{Idiomas}
\cvlanguage{Español}{Nativo}{}
\cvlanguage{Inglés}{Intermedio}{}
\cvlanguage{Japonés}{Básico}{JLPT N5}

\section{Experiencia}
\subsection{Voluntariado}
\cventry{2019-Presente}{Arch User Repository maintainer}{Arch Linux}{}{}
{
	\httplink{aur.archlinux.org/account/otreblan}
}

\section{Habilidades}
\subsection{Lenguajes de programación}
\cvlistitem{Bash}
\cvlistitem{C/C++}
\cvlistitem{C\#}
\cvlistitem{Python}
\cvlistitem{R}

\subsection{Extra}
\cvlistitem{(Neo)Vim}
\cvlistitem{CI}
\cvlistitem{Git}
\cvlistitem{Linux}
\cvlistitem{\LaTeX}
\cvlistitem{Unity}
\cvlistitem{Blender}

\section{Proyectos}
\cventry{2021}{Compilador}{\httplink{github.com/otreblan/cce}}{}{}
{Un compilador hecho en C++, C, Bison y Flex.}
\cventry{2019}{Juego}{\httplink{github.com/L603/SpaceClones}}{}{}
{Un clon de Space Invaders hecho en C++.}

\end{document}
