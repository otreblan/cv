\documentclass[10pt, a4paper]{moderncv}

\moderncvtheme[blue]{classic}

\usepackage[scale=0.81]{geometry}
\recomputelengths

\firstname{Alberto}
\familyname{Oporto Ames}
\email{alberto.oporto@utec.edu.pe}
\social[linkedin]{otreblan}
%\title{Estudiante de Ciencia de la Computación}
\mobile{915365499}
\address{Santiago de Surco, Av Las Gaviotas 2110}
\extrainfo{\githubsocialsymbol \httplink{github.com/otreblan}}

\begin{document}

\maketitle
% Ajustar la presentación de acuerdo a los requerimientos de la oferta.
% 4 líneas
\section{Presentación}
Estudio Ciencia de la Computación en la Universidad de Ingeniería y Tecnología.

Tengo interés en la tecnología y en desarrollar continuamente mis competencias tecnológicas, especialmente en proyectos Open Source.
Soy una persona autodidacta en el uso de nuevas herramientas, analizo y resuelvo problemas efectivamente.
Además, muestro escucha activa y apertura ante cualquier feedback.

\section{Educación}
\cventry{2018-Presente}{Ciencia de la Computación}{Universidad de Ingeniería y Tecnología - UTEC}{Lima}{Perú}{}

\section{Idiomas}
%\cvlanguage{Español}{Nativo}{}
\cvlanguage{Inglés}{Avanzado}{TOEFL ITP C1}
\cvlanguage{Japonés}{Básico}{JLPT N5}


\section{Lenguajes de programación}
\cvlistitem{Python}
\cvlistitem{C/C++}
\cvlistitem{C\#}
\cvlistitem{SQL}
\cvlistitem{R}
\cvlistitem{Java}

\section{Tecnologías}
\cvlistitem{Office}
\cvlistitem{Git/Github}
\cvlistitem{Linux}
\cvlistitem{Blender}

\section{Proyectos}
\cventry{2021}{Modelo predictor de matrículas}{\httplink{github.com/cs2901-2021-1/project/tree/python}}{}{}
{
	Diseñé un modelo de IA para predecir la cantidad esperada de matrículas de un curso.
	% para un determinado ciclo a partir de la información de los estudiantes habilitados en
	%una base de datos Oracle.
	Lo desarrollé con Tensorflow en Python.
}
\cventry{2021}{Compilador en C/C++ con semántica reducida}{\httplink{github.com/otreblan/cce}}{}{}
{
	Programé un compilador de una versión reducida de C en español.
	Con propósito educativo.
	Usé C++, C, Bison y Flex.
}
\cventry{2020}{Asistencia a docente}{}{}{}
{
	Apoyé en el uso de herramientas virtuales a los docentes.
	% Cómo parte de los requisitos para mantener la beca $BECA.
}
\cventry{2019}{Videojuego}{\httplink{github.com/L603/SpaceClones}}{}{}
{
	Guié a mi equipo en el desarrollo de un videojuego basado en Space Invaders,
	hecho en C++; por lo cual recibimos un reconocimiento por nuestro trabajo destacado.
}

\section{Voluntariado}
\cventry{2019-Presente}{Arch User Repository maintainer}{\httplink{aur.archlinux.org/account/otreblan}}{}{}
{
	Me encargo de actualizar, parchear y crear paquetes para la distribución
	Arch Linux.
}

%\section{Certificaciones}
% TODO: Linkedin

\end{document}
