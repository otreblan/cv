\documentclass[10pt, a4paper]{moderncv}

\moderncvtheme[blue]{classic}

\usepackage[scale=0.81]{geometry}
\recomputelengths

\firstname{Alberto}
\familyname{Oporto}
\email{alberto.oporto@utec.edu.pe}
\title{Bachiller en Ciencia de la Computación}
\mobile{+51 915365499}
\address{Lima, Perú}
\extrainfo{
	\linkedinsocialsymbol \httplink{linkedin.com/in/otreblan}\\
	\githubsocialsymbol \httplink{github.com/otreblan}
}

\begin{document}

\maketitle
% Ajustar la presentación de acuerdo a los requerimientos de la oferta.
% 4 líneas
\section{Presentación}
Desarrollador Backend con 2 años de experiencia.
Y usuario de Arch Linux desde hace 6 años.
Estudié Ciencia de la Computación en UTEC.
Tengo interés en la tecnología y en desarrollar continuamente mis competencias tecnológicas, especialmente en proyectos Open Source.
%Soy una persona autodidacta en el uso de nuevas herramientas, analizo y resuelvo problemas efectivamente.
%Además, muestro escucha activa y apertura ante cualquier feedback.

% TODO: Me falta un objetivo

\section{Experiencia}
% TODO: Potencial
\cventry{05/2023-Presente}{Ingeniero de software}{TITANIUS CONSULTING S.R.L.}{Lima}{Perú}{Me encargo de desarrollar el backend con microservicios. También he optimizado los pipelines CI para compilar y desplegar imágenes docker hacia un cluster Kubernetes.}
\cventry{02/2023-04/2023}{Programmer Trainee}{HEINZ - GLAS PERU S.A.C.}{Lima}{Perú}{Desarrollé una aplicación en C\# .NET con ASP.NET para la automatización de procesos de la planta de producción. Además escribí scripts en Bash para obtener información desde los servidores Linux.}

\section{Educación}
\cventry{2018-2024}{Ciencia de la Computación}{Universidad de Ingeniería y Tecnología - UTEC}{Lima}{Perú}{}

\section{Idiomas}
%\cvlanguage{Español}{Nativo}{}
\cvlanguage{Inglés}{Avanzado}{TOEFL ITP C1}
\cvlanguage{Japonés}{Intermedio}{JLPT N3}


\section{Lenguajes de programación}
\cvlistitem{C/C++}
\cvlistitem{Python}
\cvlistitem{Bash}
\cvlistitem{C\#}
\cvlistitem{SQL}

\section{Tecnologías}
\cvlistitem{Arduino/esp32/esp8266}
\cvlistitem{Linux}
\cvlistitem{Git/Github}

\section{Proyectos}
\cventry{2024}{Sensor de radiación uv}{\httplink{github.com/otreblan/esp8266-uv}}{}{}
{
	Con un esp8266 y un sensor de luz uv medimos la radiación a lo largo del día. Programado en C++ con Espressif RTOS.
}
\cventry{2023}{Mini supercomputadora}{\httplink{github.com/otreblan/mini-cluster}}{}{}
{
	Mi equipo y yo configuramos 7 Raspberry Pis para crear un mini cluster HPC.
	Luego ejecutamos pruebas de escalabilidad con OpenMPI.
}
\cventry{2023}{Motor de videojuegos}{\httplink{github.com/otreblan/vulkan-hello}}{}{}
{
	Programé con C++ y Vulkan un motor de videojuegos con un paradigma basado en datos (Entity Component System).
	Soporta físicas en tiempo real, input del usuario, importación de modelos 3D y renderización rasterizada.
}
%\cventry{2021}{Compilador en C/C++ con semántica reducida}{\httplink{github.com/otreblan/cce}}{}{}
%{
%	Programé un compilador de una versión reducida de C en español.
%	Con propósito educativo.
%	Usé C++, C, Bison y Flex.
%}
%\cventry{2019}{Videojuego}{\httplink{github.com/L603/SpaceClones}}{}{}
%{
%	Guié a mi equipo en el desarrollo de un videojuego basado en Space Invaders,
%	hecho en C++; por lo cual recibimos un reconocimiento por nuestro trabajo destacado.
%}

%\section{Voluntariado}
%\cventry{2019-Presente}{Arch User Repository maintainer}{\httplink{aur.archlinux.org/account/otreblan}}{}{}
%{
%	Me encargo de actualizar, parchear y crear paquetes para la distribución
%	Arch Linux.
%}

%\section{Certificaciones}
% TODO: Linkedin

\end{document}
